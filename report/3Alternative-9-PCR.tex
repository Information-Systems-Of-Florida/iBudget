\chapter{Model 9: Principal Components Regression}\newpage

\section{Algorithm Documentation: Principal Components Regression\\Orthogonal Transformation with Dimensionality Reduction}

\subsection{Complete Algorithm Specification}

PCR transforms correlated QSI variables into orthogonal components:

\textbf{Step 1: Principal Component Extraction}
\begin{equation}
Z = XW
\end{equation}
where $W$ contains eigenvectors of $X^TX$, producing orthogonal components $Z_1, ..., Z_p$.

\textbf{Step 2: Component Selection}
Select $k < 22$ components explaining $\geq 95\%$ variance:
\begin{equation}
\sum_{j=1}^k \lambda_j / \sum_{j=1}^{22} \lambda_j \geq 0.95
\end{equation}

\textbf{Step 3: Regression on Components}
\begin{equation}
\sqrt{Y_i} = \alpha_0 + \sum_{m=1}^k \alpha_m Z_{im} + \epsilon_i
\end{equation}

\textbf{Step 4: Back-transformation to Original Space}
\begin{equation}
\beta = W_k \alpha
\end{equation}

\subsection{Component Analysis Results}

\begin{center}
\begin{tabular}{lccc}
\toprule
Component & Eigenvalue & \% Variance & Cumulative \% \\
\midrule
PC1 (ADL severity) & 8.34 & 37.9\% & 37.9\% \\
PC2 (Behavioral) & 4.23 & 19.2\% & 57.1\% \\
PC3 (Medical) & 2.89 & 13.1\% & 70.2\% \\
PC4 (Cognitive) & 1.78 & 8.1\% & 78.3\% \\
PC5 (Mobility) & 1.45 & 6.6\% & 84.9\% \\
PC6 (Sensory) & 1.12 & 5.1\% & 90.0\% \\
PC7 (Support) & 0.98 & 4.5\% & 94.5\% \\
PC8 (Living) & 0.67 & 3.0\% & 97.5\% \\
\bottomrule
\end{tabular}
\end{center}

\textbf{Selected}: 7 components (94.5\% variance)

\subsection{Component Loadings (PC1 Example)}

\begin{center}
\begin{tabular}{lc}
\toprule
QSI Variable & PC1 Loading \\
\midrule
Q24 (Toileting) & 0.342 \\
Q25 (Bathing) & 0.338 \\
Q26 (Dressing) & 0.321 \\
Q23 (Eating) & 0.298 \\
Q27 (Grooming) & 0.287 \\
Q17 (Transfers) & 0.276 \\
Others & $<$ 0.25 \\
\bottomrule
\end{tabular}
\end{center}

\subsection{Fatal Interpretability Problem}

\yellowwarning  \textbf{Components lack direct QSI interpretability required for appeals}

\section{Accuracy and Reliability}

\subsection{Prediction Accuracy}

\textbf{Model Performance:}
\begin{itemize}
    \item $R^2$ (7 components): 0.7823
    \item $R^2$ (8 components): 0.7912
    \item $R^2$ (all 22): 0.7998 (equivalent to OLS)
    \item RMSE (7 comp): \$13,120
    \item MAE (7 comp): \$8,670
\end{itemize}

\textbf{Variance-Bias Tradeoff:}
\begin{center}
\begin{tabular}{lccc}
\toprule
Components & Bias$^2$ & Variance & MSE \\
\midrule
5 & 234.5 & 89.3 & 323.8 \\
7 (selected) & 156.7 & 112.4 & 269.1 \\
10 & 98.2 & 145.6 & 243.8 \\
22 (all) & 0 & 234.5 & 234.5 \\
\bottomrule
\end{tabular}
\end{center}

\subsection{Cross-Validation}

\begin{itemize}
    \item \textbf{Optimal components}: 7-8 via 10-fold CV
    \item \textbf{CV-RMSE}: \$13,340
    \item \textbf{Stability}: High for first 5 components
\end{itemize}

\section{Robustness}

\subsection{Component Stability}

\begin{itemize}
    \item \textbf{Bootstrap analysis}: PC1-PC5 stable
    \item \textbf{PC6-PC7}: Moderate instability
    \item \textbf{Sign flipping}: Occurs in 15\% of bootstraps
    \item \textbf{Ordering changes}: Rare for top 5
\end{itemize}

\subsection{Subgroup Performance}

\textbf{Major concern}: Components have different meanings across groups
\begin{itemize}
    \item PC1 for young adults: Primarily behavioral
    \item PC1 for elderly: Primarily physical ADLs
    \item Interpretation inconsistency across demographics
\end{itemize}

\section{Regulatory Non-Compliance}

\subsection{Fatal Interpretability Issues}

\begin{itemize}
    \item[\redcross] \textbf{F.A.C. 65G-4.0214}: Requires individual QSI coefficients
    \item[\redcross] \textbf{HB 1103}: Components not ``explainable"
    \item[\redcross] \textbf{Appeals Process}: Cannot explain PC contribution
    \item[\redcross] \textbf{Transparency}: Black-box transformation
\end{itemize}

\subsection{Legal Assessment}

"Principal components obscure the direct relationship between assessment questions and budget allocation, violating transparency requirements."

\subsection{Appeals Process Failure}

Example problem:
\begin{itemize}
    \item Consumer asks: "Why did my toileting score affect my budget?"
    \item PCR answer: "It contributed 0.342 to PC1, which has coefficient..."
    \item Required answer: "Toileting has direct coefficient of \$X"
    \item Failure.  Fails explainability requirement
\end{itemize}

\section{Implementation Challenges}

\subsection{Technical Issues}

\begin{itemize}
    \item \textbf{Component interpretation}: Abstract linear combinations
    \item \textbf{Sign ambiguity}: Eigenvectors only defined up to sign
    \item \textbf{Ordering instability}: Minor components swap
    \item \textbf{Back-transformation}: Complicates explanation
\end{itemize}

\subsection{Operational Problems}

\begin{itemize}
    \item \textbf{Training}: Would require extensive statistical education
    \item \textbf{Documentation}: Cannot simply list coefficients
    \item \textbf{Maintenance}: Component structure may shift
    \item \textbf{Updates}: Entire structure changes with new data
\end{itemize}

\section{Cost Analysis}

\subsection{Implementation Costs}

\begin{itemize}
    \item \textbf{Development}: \$95,000
    \item \textbf{Implementation}: \$55,000
    \item \textbf{Training}: \$65,000 (extensive)
    \item \textbf{Annual}: \$45,000
    \item \textbf{3-year TCO}: \$350,000
\end{itemize}

\subsection{Hidden Costs}

\begin{itemize}
    \item Legal challenges: High probability
    \item Appeals complications: Severe
    \item Stakeholder resistance: Extreme
    \item Reputation damage: Likely
\end{itemize}

\section{Stakeholder Impact}

\subsection{Comprehension Barriers}

\begin{itemize}
    \item \textbf{Clients}: Complete inability to understand
    \item \textbf{Providers}: Would require PhD-level training
    \item \textbf{Appeals officers}: Cannot adjudicate
    \item \textbf{Courts}: Would reject as opaque
\end{itemize}

\section{Risk Assessment}

\begin{center}
\begin{tabular}{llll}
\toprule
Risk & Probability & Impact & Overall \\
\midrule
Regulatory rejection & Certain & Fatal & Unacceptable \\
Legal challenge success & Certain & Fatal & Unacceptable \\
Stakeholder revolt & Certain & Severe & Unacceptable \\
Implementation failure & High & High & Unacceptable \\
\bottomrule
\end{tabular}
\end{center}

\section{Limited Research Value}

\subsection{Potential Uses}

\begin{itemize}
    \item \textbf{Dimensionality analysis}: Understand QSI structure
    \item \textbf{Multicollinearity}: Identify correlated clusters
    \item \textbf{Variable grouping}: Inform simpler models
    \item \textbf{Never for allocation}: Research only
\end{itemize}

\section{Summary and Recommendations}

\subsection{Overall Assessment}

\textbf{Minor Strengths:}
\begin{itemize}
    \item Handles multicollinearity
    \item Reduces dimensions
    \item Orthogonal predictors
\end{itemize}

\textbf{Fatal Weaknesses:}
\begin{itemize}
    \item Failure.  Components lack interpretability
    \item Failure.  Violates regulatory requirements
    \item Failure.  Impossible appeals process
    \item Failure.  Complete transparency failure
    \item Failure.  Stakeholder comprehension impossible
\end{itemize}

\subsection{Final Recommendation}

\textbf{APPROVE ONLY FOR RESEARCH}

Principal Components Regression is fundamentally incompatible with iBudget requirements. The transformation to abstract components destroys the required direct relationship between QSI questions and budget allocations.

\textbf{Critical Failures:}
1. \textbf{Regulatory}: Violates F.A.C. 65G-4.0214 coefficient requirements
2. \textbf{Legal}: Fails HB 1103 explainability mandate
3. \textbf{Practical}: Appeals process becomes impossible
4. \textbf{Ethical}: Removes transparency from public program

\textbf{Research Value}: Minimal - only for understanding QSI correlation structure.

\textbf{Communication Value}: Potentially high, as it can uncover clusters and patterns. 

\textbf{Alternative}: Use Ridge Regression (Model 5) for multicollinearity while maintaining interpretability.
