\chapter{Model 8: Bayesian Linear Regression}\newpage

\section{Algorithm Documentation: Bayesian Linear Regression\\Probabilistic Framework with Uncertainty Quantification}

\subsection{Complete Algorithm Specification}

Bayesian regression treats coefficients as probability distributions:

\begin{equation}
\sqrt{Y_i} = \beta_0 + \sum_{j=1}^{22} \beta_j X_{ij} + \epsilon_i
\end{equation}

With priors:
\begin{align}
\beta_j &\sim \text{Normal}(\mu_{\beta_j}, \sigma^2_{\beta_j}) \quad \text{for } j = 0, 1, ..., 22 \\
\epsilon_i &\sim \text{Normal}(0, \sigma^2) \\
\sigma^2 &\sim \text{InverseGamma}(\alpha, \beta)
\end{align}

Posterior distribution via Bayes' theorem:
\begin{equation}
p(\beta, \sigma^2 | Y, X) \propto p(Y | X, \beta, \sigma^2) \cdot p(\beta) \cdot p(\sigma^2)
\end{equation}

\subsection{Prior Specification}

\textbf{Informed priors based on Model 5b:}
\begin{itemize}
    \item \textbf{Coefficient means}: $\mu_{\beta_j}$ = Model 5b estimates
    \item \textbf{Coefficient variance}: $\sigma^2_{\beta_j}$ = 2 $\times$ Model 5b SE$^2$
    \item \textbf{Error variance}: $\alpha = 3$, $\beta = 2\sigma^2_{OLS}$
\end{itemize}

\subsection{Input Variables}

All 22 QSI predictors with posterior distributions:
\begin{enumerate}
    \item Each predictor has posterior $p(\beta_j | \text{Data})$
    \item Full posterior covariance matrix $\Sigma_{\beta}$
    \item Uncertainty propagated through predictions
\end{enumerate}

\subsection{Output Specification}

\textbf{Posterior predictive distribution:}
\begin{equation}
p(\tilde{Y}_i | X_i, \text{Data}) = \int p(\tilde{Y}_i | X_i, \beta, \sigma^2) \cdot p(\beta, \sigma^2 | \text{Data}) d\beta d\sigma^2
\end{equation}

\textbf{Point estimates and intervals:}
\begin{itemize}
    \item Mean: $\mathbb{E}[\text{Budget}_i] = \mathbb{E}[(\tilde{Y}_i)^2]$
    \item Median: $\text{Median}[(\tilde{Y}_i)^2]$
    \item 95\% Credible Interval: $[\text{Budget}_{0.025}, \text{Budget}_{0.975}]$
\end{itemize}

\subsection{Fatal Regulatory Flaw}

Medicaid requires deterministic budgets, not probability distributions. Florida Statute 393.0662 mandates a single allocation amount, making probabilistic outputs legally impossible.

\section{Accuracy and Reliability}

\subsection{Prediction Accuracy}

\textbf{Point Estimate Performance:}
\begin{itemize}
    \item Posterior mean $R^2$: 0.8012
    \item RMSE: \$12,340
    \item MAE: \$8,190
    \item DIC: 158,234
    \item WAIC: 158,456
\end{itemize}

\textbf{Uncertainty Calibration:}
\begin{center}
\begin{tabular}{lcc}
\toprule
Credible Interval & Nominal Coverage & Actual Coverage \\
\midrule
50\% & 50\% & 51.2\% \\
80\% & 80\% & 79.8\% \\
95\% & 95\% & 94.6\% \\
99\% & 99\% & 98.9\% \\
\bottomrule
\end{tabular}
\end{center}

\subsection{Posterior Distributions}

\textbf{Coefficient Uncertainty:}
\begin{center}
\begin{tabular}{lccc}
\toprule
Predictor & Posterior Mean & Posterior SD & 95\% CI Width \\
\midrule
Behavioral (Q30) & 67.8 & 4.32 & 17.1 \\
Medical (Q29) & 28.9 & 2.14 & 8.4 \\
ADL composite & 78.4 & 5.67 & 22.3 \\
Living setting & 45.2 & 3.89 & 15.3 \\
\bottomrule
\end{tabular}
\end{center}

\subsection{MCMC Convergence}

\begin{itemize}
    \item \textbf{Chains}: 4 parallel chains, 10,000 iterations each
    \item \textbf{Burn-in}: 2,000 iterations
    \item \textbf{Thinning}: Every 5th iteration retained
    \item \textbf{Gelman-Rubin $\hat{R}$}: All parameters $<$ 1.01
    \item \textbf{Effective sample size}: $>$ 4,000 for all parameters
\end{itemize}

\section{Robustness}

\subsection{Prior Sensitivity Analysis}

\begin{center}
\begin{tabular}{lccc}
\toprule
Prior Type & Posterior Mean $R^2$ & CI Width & Shrinkage \\
\midrule
Informative (Model 5b) & 0.8012 & Narrow & Moderate \\
Weakly informative & 0.7998 & Medium & Low \\
Non-informative & 0.7991 & Wide & None \\
Skeptical & 0.7923 & Narrow & High \\
\bottomrule
\end{tabular}
\end{center}

\subsection{Model Averaging}

\begin{itemize}
    \item \textbf{BMA over transformations}: Log vs sqrt
    \item \textbf{Posterior model probabilities}: Sqrt (0.62), Log (0.38)
    \item \textbf{Averaged predictions}: Improved calibration
\end{itemize}

\subsection{Stress Testing}

\begin{itemize}
    \item \textbf{Prior-data conflict}: Detected via posterior predictive checks
    \item \textbf{Influence analysis}: No single observation dominates
    \item \textbf{Cross-validation}: LOO-CV shows stable performance
\end{itemize}

\section{Sensitivity to Outliers and Missing Data}

\subsection{Robust Bayesian Extensions}

\begin{itemize}
    \item \textbf{Student-t errors}: $\epsilon_i \sim t_\nu(0, \sigma^2)$
    \item \textbf{Degrees of freedom}: $\nu \sim \text{Gamma}(2, 0.1)$
    \item \textbf{Automatic outlier accommodation}: Via heavy tails
    \item \textbf{Coverage}: 100\% of observations
\end{itemize}

\subsection{Missing Data}

\begin{itemize}
    \item \textbf{Natural handling}: Missing data as parameters
    \item \textbf{Joint inference}: Data and parameters together
    \item \textbf{No exclusions}: Full sample retained
    \item \textbf{Imputation uncertainty}: Propagated through posterior
\end{itemize}

\section{Implementation Feasibility}

\subsection{Technical Requirements}

\begin{itemize}
    \item \textbf{Software}: Stan, JAGS, PyMC3, brms
    \item \textbf{Computation}: 30-60 seconds for full posterior
    \item \textbf{Memory}: 2GB for posterior samples
    \item \textbf{Hardware}: GPU acceleration beneficial
    \item \textbf{Storage}: 500MB per model with full posterior
\end{itemize}

\subsection{Operational Barriers}

\begin{itemize}
    \item Cannot provide single allocation required by law
    \item Probability distributions not allowed in regulations
    \item Appeals cannot handle uncertainty ranges
    \item Staff would require PhD-level Bayesian training
    \item Research and validation only
\end{itemize}

\section{Complexity, Cost, Resources, and Regulatory Alignment}

\subsection{Technical Complexity}

\begin{itemize}
    \item[\yellowwarning] \textbf{Mathematical sophistication}: Very high
    \item[\yellowwarning] \textbf{Interpretability}: Requires statistical expertise
    \item[\yellowwarning] \textbf{Maintenance}: Complex MCMC diagnostics
    \item[\yellowwarning] \textbf{Updates}: Full posterior re-estimation
\end{itemize}

\subsection{Cost Analysis}

\begin{itemize}
    \item \textbf{Development}: \$145,000 (specialized expertise)
    \item \textbf{Infrastructure}: \$65,000 (computing resources)
    \item \textbf{Training}: \$55,000 (Bayesian concepts)
    \item \textbf{Annual}: \$75,000 (maintenance)
    \item \textbf{3-year TCO}: \$490,000
\end{itemize}

\subsection{Regulatory Non-Compliance}

\begin{itemize}
    \item[\redcross] \textbf{F.S. 393.0662}: Requires deterministic amount
    \item[\redcross] \textbf{F.A.C. 65G-4.0214}: No probabilistic provisions
    \item[\redcross] \textbf{HB 1103}: Posterior distributions not "explainable"
    \item[\redcross] \textbf{CMS}: Budget must be fixed, not range
    \item[\redcross] \textbf{Due Process}: Cannot appeal probability
\end{itemize}

\subsection{Legal Assessment}

Bayesian methods produce probability distributions fundamentally incompatible with statutory requirement for single, deterministic budget amounts. Implementation would violate Florida law.

\section{Adaptability and Maintenance}

\subsection{Dynamic Updates}

\begin{itemize}
    \item \textbf{Prior updating}: Sequential Bayesian learning
    \item \textbf{Online learning}: Real-time posterior updates
    \item \textbf{Hyperparameter tuning}: Empirical Bayes
    \item \textbf{Model expansion}: Natural framework for complexity
\end{itemize}

\subsection{Monitoring Requirements}

\begin{itemize}
    \item \textbf{Convergence diagnostics}: Every run
    \item \textbf{Posterior predictive checks}: Monthly
    \item \textbf{Prior-posterior overlap}: Quarterly
    \item \textbf{Model comparison}: DIC, WAIC tracking
\end{itemize}

\section{Stakeholder Impact and Acceptance}

\subsection{Comprehension Barriers}

\begin{itemize}
    \item \textbf{Clients}: Would not understand distributions
    \item \textbf{Staff}: Requires statistical sophistication
    \item \textbf{Legal}: Incompatible with framework
    \item \textbf{Political}: Appears indecisive/uncertain
    \item \textbf{Public}: Loss of trust in "uncertain" allocations
\end{itemize}

\subsection{Communication Challenges}

\begin{itemize}
    \item Cannot explain "Your budget is probably between X and Y"
    \item Credible intervals meaningless to consumers
    \item Appeals impossible with probabilistic allocations
    \item Media would portray as government uncertainty
\end{itemize}

\section{Risk Assessment and Mitigation}

\begin{center}
\begin{tabular}{llll}
\toprule
Risk Category & Probability & Impact & Status \\
\midrule
Regulatory rejection & Certain & Fatal & Blocked \\
Stakeholder confusion & Certain & High & Blocked \\
Implementation complexity & High & High & Manageable \\
Research value loss & Low & Medium & Acceptable \\
Legal challenge & Certain & Fatal & Blocked \\
\bottomrule
\end{tabular}
\end{center}

\section{Performance Monitoring Plan}

\subsection{Bayesian Diagnostics}

\begin{itemize}
    \item \textbf{Convergence}: $\hat{R} < 1.01$ all parameters
    \item \textbf{ESS}: $>$ 1000 per parameter
    \item \textbf{Posterior predictive}: p-values centered
    \item \textbf{Prior-posterior overlap}: Moderate
    \item \textbf{MCMC efficiency}: $> 0.1$
\end{itemize}

\subsection{Quality Metrics}

\begin{itemize}
    \item \textbf{Calibration}: Coverage probability accuracy
    \item \textbf{Sharpness}: Prediction interval width
    \item \textbf{Bias}: Posterior mean vs actual
    \item \textbf{Information criteria}: DIC, WAIC trends
\end{itemize}

\section{Research Value}

\subsection{Valid Applications}

\begin{itemize}
    \item \textbf{Uncertainty quantification}: Know precision of estimates
    \item \textbf{Risk assessment}: Identify high-uncertainty cases
    \item \textbf{Policy simulation}: Posterior predictive checks
    \item \textbf{Model validation}: Compare with frequentist
    \item \textbf{Decision support}: Not decision making
\end{itemize}

\subsection{Parallel Analysis Benefits}

\begin{itemize}
    \item Coefficient stability assessment
    \item Prediction interval validation
    \item Prior-data conflict detection
    \item Outlier identification via posterior
    \item Model comparison framework
\end{itemize}

\section{Summary and Recommendations}

\subsection{Overall Assessment}

\textbf{Strengths (Research):}
\begin{itemize}
    \item Complete uncertainty quantification
    \item Principled handling of all uncertainty sources
    \item Natural missing data accommodation
    \item Rich inferential framework
    \item Coherent probability statements
\end{itemize}

\textbf{Fatal Weaknesses (Production):}
\begin{itemize}
    \item Cannot produce required single allocation
    \item Probability distributions violate all regulations
    \item Incompatible with appeals process
    \item Would require complete statutory rewrite
    \item Stakeholder comprehension insurmountable
\end{itemize}

\subsection{Final Recommendation}

\textbf{REJECT for Budget Allocation}

\textbf{APPROVE for Research Only}

Bayesian regression is fundamentally incompatible with Florida's deterministic budget requirement. The probabilistic nature of Bayesian inference cannot be reconciled with current law requiring a single allocation amount.

\textbf{Research Implementation Strategy:}
\begin{itemize}
    \item Deploy as uncertainty quantification tool
    \item Use for coefficient stability analysis
    \item Support model validation efforts
    \item Inform policy stress testing
    \item Never use for actual allocations
\end{itemize}

\textbf{Future Consideration:} If regulations evolve to allow uncertainty ranges or probabilistic allocations, Bayesian methods would provide the most principled framework for implementation.

\textbf{Critical Point:} Any attempt to use Bayesian methods for actual budget allocation would immediately violate F.S. 393.0662 and trigger legal action. The method's value lies exclusively in research and validation applications.
 