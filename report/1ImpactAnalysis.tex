\chapter{Impact Analysis}\newpage

\section{Introduction}

The Florida iBudget algorithm represents a critical component of the state's developmental disability services infrastructure, determining individual budget allocations for Home and Community-Based Services (HCBS) under the Developmental Disabilities Individual Budgeting waiver program. This system currently serves over 36,000 enrollees, making algorithmic decisions that directly impact the quality of life and service access for individuals with developmental disabilities across Florida. The algorithm's role extends beyond mere budget calculation; it fundamentally shapes how resources are distributed, what services individuals can access, and how person-centered planning principles are implemented in practice.

The enactment of House Bill 1103 in the 2025 legislative session has fundamentally altered the regulatory landscape for iBudget allocation methodologies. This legislation mandates a comprehensive study to review, evaluate, and identify recommendations regarding the current algorithm, with particular emphasis on ensuring compliance with person-centered planning requirements under section 393.0662, Florida Statutes. The bill's requirements extend beyond simple algorithmic refinement, demanding a fundamental reassessment of how statistical methods align with person-centered planning principles and contemporary disability services philosophy.

This analysis addresses the impact of the iBudget recommendations detailed in the [======] Report. 

