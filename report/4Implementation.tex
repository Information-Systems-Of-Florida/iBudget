%%%%%%%%%%%%%%%%%%%%%%%%%%%%%%%%%%%%%%%%%%%%%%%%%%%%%%%%%%%%%%%%%
\chapter{Algorithm Comparison and Selection}  \newpage
%%%%%%%%%%%%%%%%%%%%%%%%%%%%%%%%%%%%%%%%%%%%%%%%%%%%%%%%%%%%%%%%%

The 2015 iBudget algorithm public comments revealed critical issues that remain highly relevant for today's algorithm development. Key concerns included the algorithm's failure to properly account for transportation costs (which varied from \$6-30 per trip regionally), harsh impacts on people living in family homes due to inadequate caregiver variables, and the QSI assessment tool's inability to capture complex medical conditions, aging-related decline, and gradual physical deterioration. Stakeholders emphasized that algorithms must achieve equitable funding distribution among people with similar needs rather than just cost containment, and that certain services with high regional variability should be handled outside the main algorithm. The comments also highlighted the need for transparent, explainable methodologies that support the appeals process, more frequent assessments (every 3 years as intended, not 5+), and consideration of provider network sustainability given the 14.17\% rate reduction since 2003.

These historical lessons directly inform modern algorithm requirements by establishing that successful implementation requires balancing statistical sophistication with practical realities. Today's algorithms should use validated assessment tools that comprehensively capture aging, behavioral, medical, and caregiver factors; employ robust statistical methods that handle outliers transparently rather than through arbitrary exclusion; maintain interpretability for appeals processes (providing strong guidance for ``black box" approaches like neural networks); and undergo phased implementation with pilot testing and stakeholder education. The 2015 experience demonstrates that purely statistical optimization without considering stakeholder understanding, systemic sustainability, and operational feasibility leads to problematic outcomes that undermine both individual equity and system-wide effectiveness.