\documentclass[12pt, letterpaper]{article}
\usepackage{graphicx}
\usepackage{hyperref}

\begin{document}

\thispagestyle{empty}
\begin{center}
	{\large \textbf{{APD Statistical Validation and Documentation}}} 


	\begin{figure}[th]
		\centering
		\includegraphics[width=10cm]{../assets/APD-logo-BG.png}
	\end{figure}

	\vspace{\fill}
	\begin{figure}[th]
		\centering
		\includegraphics[width=3cm]{../assets/ISF-logo-medium.png}
	\end{figure}

	\vspace{\fill}



	{\normalsize \today}


	

\end{center}

\newpage

\thispagestyle{empty}

\begin{flushleft}

\phantom{ghost text}

\vspace{\fill}

\noindent \textbf{Agency for Persons with Disabilities (APD)} \\
iBudget Algorithm Study\\
Procurement Office: Tamara Harrington \\
4030 Esplanade Way \\
Tallahassee, Florida 32399

\vspace{1cm}
\noindent \textbf{ISF, Inc.} \\
\url{https://www.isf.com}\\
Report prepared by \href{mailto:akunerth@isf.com} {Allison K.Kunerth, Ph.D.}\\
ORCID Researcher Number \href{https://orcid.org/0000-0002-9035-507X}{0000-0002-9035-507X}\\
Senior Management Consultant and Data Scientist, ISF, Inc.\\
\today\\

\vspace{\fill}
\noindent \textbf{Full Study prepared by ISF, Inc.} \\
 Annika Baeten, Management Consultant \\
Abbie David, PMP, PhD, Project Manager\\
 Jason Dillaberry, Client Partner \\
 Matthew Fisher, JD, LLM, MBA, Principal Consultant \\
 Juan B. Gutiérrez, PhD, Mathematician \\
 Jessica Kemper, MPA, Senior Management Consultant \\
 Chris Klass, MPP Finance, Senior Management Consultant \\
Allison Kunerth, PhD, Senior Management Consultant \\
 Daniel Margolis, Associate Management Consultant \\
\end{flushleft}




%********** End Frontmatter **********
%%The following loads the picture on top of every page, the numbers in \put() define the position on the page:
%\AddToShipoutPicture{\setlength\unitlength{0.1mm}\put(604,2522){\includegraphics[width=1.7cm]{Logo-Approved2.0-DPIL-ILSA-All-CAPS-150x150}}}

\pagestyle{cb} % now we want to have headers and footers

\newpage

\tableofcontents

\newpage

\section{Feature Selection} 
Model Validation is being completed by Juan,Daniel and Allison. This document serves to track this process as much as possible given the time constraints.\\

Confirmed that outlier removal configuration is set to false.\\
Confirmed random seed set to 42.\\
Confirmed Feature Selection Analysis used Numpy.\\

Variables checking shows the following are missing (i.e., not explicitly called into the programming--they may exist in one of the functions I have yet to check, and/or be recoded.):\\
\begin{itemize}\\
\noindent \textbf{Possibly already included, or may need to be considered.}
	\item RH1
	\item RH2
	\item RH3
	\item RH4
	\item Sex - May be Gender, need to see if recoded
	\item CNTY RESID NAME -May be captured in recoded County variable
	\item CurrentAge - Need to check how age is accounted for
	\item AGE50P
	\item AGE60P
	\item AGE70P
	\item FuncationalStatus
	\item BehavioralStatus
	\item PhysicalStatus
	\item EstLevel - Maybe this is OLEVEL?
	\item FunctSum - likely recoded to FSUM
	\item BehavSum - likely recoded to BSUM
	\item PhysSum - likely revoded to PSUM
	\item Days Since QSI
	\item Live2ILSL
	\item Live2RH1
	\item Live2RH2
	\item Live2RH3
	\item Live2RH4
	\item Age21 to 30
	\item Age 30 plus
	\item BSum
	\item FHFSum
	\item SLFSum
	\item SLBSum
	\item iConnect Category
	\item AlgorithmAmtModel5b
	\item ClientLivingSetting\\
	
\noindent \textbf{Other Items Not included (i.e., not called explicitly)}\\
	\item CaseNo
	\item FSNumFiscalYears
	\item BeginDate
	\item ENDDATE
	\item IndexSubObjectCode
	\item ConsumerCounty
	\item GeographicDifferential
	\item ProviderRateType
	\item PROC
	\item Service
	\item UnitType
	\item UnitsPer
	\item UnitsOfMeasure
	\item TotalUnits
	\item VendorId
	\item ProviderName
	\item ProviderMedcId
	\item Rate
	\item TotalAmount
	\item PlannedServiceId
	\item PlanId
	\item PIN
	\item SSN
	\item SBPG C1
	\item Region2
	\item MedicaidId
	\item LNAME
	\item FNAME
	\item MI
	\item SFX
	\item ConsumerID
	\item District
	\item Region
	\item Usable
	\item FY_StartDate
	\item FY_EndDate
	\item QSI_ApprovalDate
	\item FirstServiceDate
	\item LastServiceDate
	\item DOB
	\item MEDC_ID
	\item CNTY_RECMD
	\item CNTY_RESID
	\item REGION_NAME
	\item PrimDisabilityDescription
	\item SecondaryDisabilityDescription
	\item OtherDisabilityDescription
	\item Status PrimDisabilityDescription
	\item Category
	\item GroupedProgCompDescript
	\item ProgCompDesc
	\item Worker Dist
	\item Worker SD
	\item Worker Unit
	\item Worker Code
	\item Worker First Name
	\item Worker Last Name
	\item Worker SSN
	\item WSC filtered
	\item WL Priority
	\item DUPLICATE
	\item CLIENTID
	\item CompletedDate
	\item Q51A
	\item Q13a
	\item Q13b
	\item Q13c
	\item RATERID
	\item ResHabFlag
	\item FamilyHome
	\item IndLiving_SuppLvg
	\item ResHabSrvcPlan_notResHabLivSetting Flag
	\item Intercept
	\item Q_16
	\item Q_18
	\item Q_20
	\item Q_21
	\item Q_23
	\item Q_28
	\item Q_33
	\item Q_34
	\item Q_36
	\item Q_43
	\item CoefficientsSum
	\item CALCULATE
	\item InvalidAlgorithmLivingSetting
\end{itemize}\\
	

\subsection{Section 2.2} 
Analysis encompasses 6 fiscal years of data. Verified that data begins September 1st, 2019 and ends August 31st, 2025. Accordingly, 6- years is correct. 
\subsection{Section 2.3.1} 
Exlusions seem reasonable, including those with zero variance since they would have no impact on the outcome-and is better than excluding just low variance. However, charts exlude samples with a Total Cost < 0 due to original outlier removal.

The report has the following values, and those with differences are noted. 
\begin{itemize}
     \item Initial records min = 42,677 - Confirmed in Logs
     \item Initial records max = 47,797- Confirmed in Logs
     \item Initial records mean ~ 45,527
     \item Post-Filtering min = 41,570 - Confirmed in Logs
     \item Post-Filterin max = 47,337 - Confirmed in Logs
     \item Analysis-eligible min = 29,566 - Confirmed in Logs
     \item Analysis-eligible max = 35,329 - Confirmed in Logs
	 \item MidYear QSI = 957, 0.5\%
	 \item Late Entry = 43,102, 20.2\%
	 \item Early Exit = 14,873, 7.0\%
	 \item No Costs = 109, 0.1\%
	 \item Negative Costs = 0, 0.0\%
	 \item Insufficient Service = 81,901, 38.5\%
	 \item No QSI = 128,791, 60.5\%
\end{itemize}

\paragraph{Question 1} Do we need to include the cost requirement greater than 0 since no cases meet that definition?
\paragraph{Answer 1} That weas filtered from data preparation for the cache. Look at sql transformation 14 0r 13

\paragraph{Question 2} Cardinality is number of levels of each variable being considered. Need to check whether the number 50 is correct.
\paragraph{Answer 2} Categorical variables with more than 50 unique levels are excluded to avoid unreliable chi-square estimates and extreme sparsity.The value 50 comes from the approximate number of `valid' QSI categorical indicators which is about 52 total items − 2 summary variables = 50.
\paragraph{Question 3} Where is'OLevel' defined/derived? I know it is the ordinal mapping of the variables, but didn't immediately see it.
\paragraph{Answer 3}OLEVEL originates from the QSI Assessments tables 

tbl_QSIAssessments / tbl_QSIAssessmentsLegacy

Represents the Overall Level of need computed from the other QSI domains. It is imported into analysis as a numeric/ordinal feature after label-encoding.

\subsection{Section 2.3.2} Items Checked or Confirmed:\\
\begin{itemize}
	\item Feature Mask is accounted for  starting at line 599 with the Build mask code chunk. Appears to assign a missing value for anything that is not a finite positive number, then applies to every column. \bf{Note: Once I can get the script running again I can run this chunk and verify that the `return arrs' yields the expected results}

\end{itemize}

Questions or To-Do's\\
\begin{itemize}
	\item Validate for MISSING value applied correctly
	\item Validate label encoding for OLEVEL
	\item Validate appropriateness of <25 for rare level consolidation
	\item Validate type conversion to float64
	\item Validate VIF/Z-Score normalization - \textbf{I saw in one of the docs something about recommending that this occur instead of stating that it was done. So should check for consistency at some point}
\end{itemize}
\paragraph{Question 4} Is median imputation applied by variable or by case? I am assuming it is by variable (e.g., the median value for that variable.)
\paragraph{Answer 4} TBD

\subsection{Section 2.12}


7. Analyst Checklist\\
Log shows the following counts for each year. \bf{Note that the report asks to verify that pre-filter, filrered, and final counts are non-decreasing. However, we should expect them to decrease within each year as more cases are filtered out, and is exhibited below.}
\begin{itemize}
	\item 2025 Total Records = 48,797
	\item 2025 Records After Quality Filters = 47,337
	\item 2025 Final with Total Cost greater than 0 = 35,329
	\item 2024 Total Records = 47,259
	\item 2024 Records After Quality Filters = 46,269
	\item 2024 Final with Total Cost greater than 0 = 	34,034
	\item 2023 Total Records = 46,193
	\item 2023 Records After Quality Filters = 45,359 
	\item 2023 Final with Total Cost greater than 0 = 33,001
	\item 2022 Total Records = 45,270
	\item 2022 Records After Quality Filters = 44,090
	\item 2022 Final with Total Cost greater than 0 = 32,107
	\item 2021 Total Records = 43,968
	\item 2021 Records After Quality Filters = 42,779
	\item 2021 Final with Total Cost greater than 0 = 30,738
	\item 2020 Total Records = 42,677
	\item 2020 Records After Quality Filters = 41,570
	\item 2020 Final with Total Cost greater than 0 = 29,566
\end{itemize}

Confirmation of four figures per FY without missing/empty axes or irregularities showed:
\begin{itemize}
	\item Four plots were produced for each of the 6-years.
	\item 2021-2025 pairplot present, showed top features as LOSRI, Total Cost and Age. \bf{2020 only showed Total Cost and Age.}
	\item All years for the mixed correlation ratio are \bf{missing data for Q44 and Gender.}
	\item All years for the continuous spearman have an extra column for LOSRI by Age.
	\item All years for thecategorical cramers have an extra column for OLEVEL by Gender.
\end{itemize}



Log shows that 6 features were dropped. \bf{MI configurations were not explicit in the log and there were no small-sample notices of n less than 30.}
\begin{itemize}
	\item BSum
	\item FSum
	\item PSum
	\item Q14
	\item Q25
	\item Q32	
\end{itemize}

Log shows that 3 features had high cardinality. 
\begin{itemize}
	\item County
	\item SecondaryDiagnosis
	\item OtherDiagnosis	
\end{itemize}
Log shows the following redundancy features dropped.
\begin{itemize}
	\item 2025 pair plot top features = OLEVEL
	\item 2024 pair plot top features = LOSRI
	\item 2023 pair plot top features = LOSRI	
	\item 2022 pair plot top features = OLEVEL
	\item 2021 pair plot top features = OLEVEL
	\item 2020 pair plot top features = OLEVEL
\end{itemize}

Cross-check of TopFeaturesTable.tex post pruning showed the following.\\ Note that the ranking in the table is different than in the logs due to apparent sorting by Mean MI
\begin{itemize}
	\item 6 Years in Top 10 and 20 = Residence, Leving Setting, Age, AgeGroup, County, BLEVEL, Q26, and Q36
	\item 5 years in Top 10 and 6 years in Top 20 = Q27
	\item 3 years in Top 10 and 20 = LOSRI and OLEVEL
	\item \end{itemize}
\bf{items not showing in log, but showing in TopFeaturesTable}
\begin{itemize}
	\item 1 year in Top 10 and 6 years in Top 20 = Q20
	\item 0 years in Top 10 and 6 years in Top 20 = FLEVEL, PLEVEL, Q44,Q30,Q23, Q28, Q29
	\item 0 years in Top 10 and 4 years in Top 20 = Q21, and Q47
	\item 0 years in Top 10 and 2 years in Top 20 = Q19 and Q16
\end{itemize}
\section{Model 1:} 
TBD


\section{Model 2:}
TBD


\section{Model 3:}
TBD


\section{Model 4:} 
TBD


\section{Model 5:} 
TBD


\section{Model 6:} 
TBD


\section{Model 7:}
TBD


\section{Model 8:} 
TBD


\section{Model 9:} 
TBD


\section{Model 10:} 
TBD

    `

\end{document}