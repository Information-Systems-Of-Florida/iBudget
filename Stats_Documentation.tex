\documentclass[12pt, letterpaper]{article}
\usepackage{graphicx}
\usepackage{hyperref}
\author{Allison K. Kunerth, Ph.D}
\begin{document}

\thispagestyle{empty}
\begin{center}
	{\large \textbf{{APD Statistical Validation and Documentation}}} 

	\begin{figure}[th]
		\centering
		\includegraphics[width=10cm]{../assets/APD-logo-BG.png}
	\end{figure}

	\vspace{\fill}
	\begin{figure}[th]
		\centering
		\includegraphics[width=3cm]{../assets/ISF-logo-medium.png}
	\end{figure}

	\vspace{\fill}



	{\normalsize \today}


	

\end{center}

\newpage

\thispagestyle{empty}

\begin{flushleft}

\phantom{ghost text}

\vspace{\fill}

\noindent \textbf{Agency for Persons with Disabilities (APD)} \\
iBudget Algorithm Study\\
Procurement Office: Tamara Harrington \\
4030 Esplanade Way \\
Tallahassee, Florida 32399

\vspace{1cm}
\noindent \textbf{ISF, Inc.} \\
\url{https://www.isf.com}\\
Report prepared by \href{mailto:akunerth@isf.com} {Allison K.Kunerth, Ph.D.}\\
ORCID Researcher Number \href{https://orcid.org/0000-0002-9035-507X}{0000-0002-9035-507X}\\
Senior Management Consultant and Data Scientist, ISF, Inc.\\
\today\\

\vspace{\fill}
\noindent \textbf{Full Study prepared by ISF, Inc.} \\
 Annika Baeten, Management Consultant \\
Abbie David, PMP, PhD, Project Manager\\
 Jason Dillaberry, Client Partner \\
 Matthew Fisher, JD, LLM, MBA, Principal Consultant \\
 Juan B. Gutiérrez, PhD, Mathematician \\
 Jessica Kemper, MPA, Senior Management Consultant \\
 Chris Klass, MPP Finance, Senior Management Consultant \\
Allison Kunerth, PhD, Senior Management Consultant \\
 Daniel Margolis, Associate Management Consultant \\
\end{flushleft}




%********** End Frontmatter **********
%%The following loads the picture on top of every page, the numbers in \put() define the position on the page:
%\AddToShipoutPicture{\setlength\unitlength{0.1mm}\put(604,2522){\includegraphics[width=1.7cm]{Logo-Approved2.0-DPIL-ILSA-All-CAPS-150x150}}}

\pagestyle{cb} % now we want to have headers and footers

\newpage

\tableofcontents

\newpage

\section{Feature Selection} 
Model Validation is being completed by Juan,Daniel and Allison. This document serves to track this process as much as possible given the time constraints.
\subsection{Section 2.2} 
Analysis encompasses 6 fiscal years of data. End of FY is August 31st, 2025. Accordingly, 6- years is correct. 
\subsection{Section 2.3.1} 
Exlusions seem reasonable, including those with zero variance since they would have no impact on the outcome-and is better than excluding just low variance. 
\begin{itemize}
    \item MidYear QSI = 957, 0.5\%
    \item Late Entry = 43,102, 20.2\%
    \item Early Exit = 14,873, 7.0\%
    \item No Costs = 109, 0.1\%
    \item Negative Costs = 0, 0.0\%
    \item Insufficient Service = 81,901, 38.5\%
    \item No QSI = 128,791, 60.5\%
    \item Need to double check that the min, max and mean number of records is correct
\end{itemize}

\paragraph{Question 1} Do we need to include the cost requirement greater than 0 since no cases meet that definition?
\paragraph{Answer 1} That weas filtered from data preparation for the cache. Look at sql transformation 14 0r 13

\paragraph{Question 2} Cardinality is number of levels of each variable being considered. Need to check whether the number 50 is correct.
\paragraph{Answer 2} Categorical variables with more than 50 unique levels are excluded to avoid unreliable chi-square estimates and extreme sparsity.The value 50 comes from the approximate number of `valid' QSI categorical indicators about52 total items − 2 summary variables = 50.
\paragraph{Question 3} Where is'OLevel' defined/derived? I know it is the ordinal mapping of the variables, but didn't immediately see it.
\paragraph{Answer 3}OLEVEL originates from the QSI Assessments tables 

tbl_QSIAssessments / tbl_QSIAssessmentsLegacy

Represents the Overall Level of need computed from the other QSI domains. It is imported into analysis as a numeric/ordinal feature after label-encoding.
\section{Model 1:} 
TBD


\section{Model 2:}
TBD


\section{Model 3:}
TBD


\section{Model 4:} 
TBD


\section{Model 5:} 
TBD


\section{Model 6:} 
TBD


\section{Model 7:}
TBD


\section{Model 8:} 
TBD


\section{Model 9:} 
TBD


\section{Model 10:} 
TBD

    `

\end{document}